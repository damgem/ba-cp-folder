% This is the config file... have fun editing

% Type of thesis: dissertation,master,bachelor,seminar,report or custom
% If none of the above meets your requirement, set this value to "custom" and
% just enter the type of this document for "\thesiscustomtype". It will be
% printed on the title page
\newcommand{\thesistype}{custom}
\newcommand{\thesiscustomtype}{Guideline}

% Insert your name
\newcommand{\thesisauthor}{Arnold Sykosch}

% Insert matrikel number here.
\newcommand{\thesismatrikel}{127 0 0 1}

% Give your work a title
\newcommand{\thesistitle}{Guideline for the Composition of Master Theses, Seminar Papers and Lab Reports}
\newcommand{\thesissubtitle}{A Real Good Example Text for a Template}

% Insert your degree course like 'Computer Science (B.Sc.)' or 'Informatik (B.Sc.)'
\newcommand{\thesisdegreecourse}{Informatik (B.Sc.) \& Informatik (M.Sc.)}
% The date of completion
% e.g. {September~2014}
\newcommand{\thesisubmission}{Bonn, \today}

% The three affiliation definitions below can be left empty
% and will default to 'Universität Bonn' or 'University of Bonn'
% Configure your first and second examiner as well as your supervisor (if applicable, can be left empty to omit)
\newcommand{\thesissupervisorone}{Prof. Dr.~Michael Meier}
\newcommand{\thesissupervisoroneaffiliation}{}
\newcommand{\thesissupervisortwo}{Dr.~Matthias Frank}
\newcommand{\thesissupervisortwoaffiliation}{}
\newcommand{\thesissponsor}{Arnold Sykosch, M.Sc.}
\newcommand{\thesissponsoraffiliation}{}

\newcommand{\thesisaffiliation}{%
Rheinische Friedrich-Wilhelms-Universität Bonn\\
Institut für Informatik IV\\
Arbeitsgruppe für IT-Sicherheit\\
}

% If you would like to print your thesis twosided say twoside,
% otherwise leave empty
\newcommand{\thesisprintingstyle}{}

% BCOR binding correction.
% The space in the *inner* side of the page.
% http://www.khirevich.com/latex/page_layout
% hint: to have symetric two sided print use 20mm
\newcommand{\thesisbcor}{0cm}

% language may be \english or \german
\newcommand{\thesislanguage}{english}

% The thesis template has color accents in the color
% of the University. If you rather prefer (for whatever reason)
% a monochrome setting, insert true, else leave empty
\newcommand{\thesismonochrome}{}

% If you want to have some kind of acknowledgement set to "true"
% and place a file "thanks.tex" into the chapter folder
% If you don't want to give acknowledgements, leave empty
\newcommand{\thesisthankyou}{true}

% If you want to have an abstract set to "true"
% If you don't want to have an abstract, leave empty
% Should only be necessary for anything longer than single-digit page numbers, i.e. theses
\newcommand{\thesisabstract}{false}

% If you want to have a table of contents set to "true"
% If you don't want to have a table of contents, leave empty
% Should only be necessary for anything longer than single-digit page numbers, i.e. theses
\newcommand{\thesistableofcontents}{true}

% Here you can select where is the position of your list of figures and your list of tables
% If you want it after the table of content type 'beginning'.
% If you want it after your bibliography type 'end'.
% If you don't need this tables then leave it empty.
\newcommand{\thesistableposition}{end}

% If you want to have a declaration of ownership, change this to "true"
% Should only be necessary for theses (dissertation,master,bachelor)
\newcommand{\thesisprintdeclaration}{true}

% If you would like to print a glossary at the end,
% set to "true", otherwise leave empty.
\newcommand{\thesisprintglossary}{true}

% There are different styles for references.
% The standard reference format ist [MWM18]
% If you want only numbers in the brackets like [1] then say 'number'.
\newcommand{\thesisreferencestyle}{}
