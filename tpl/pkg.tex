% ------------------------------------------------  Needed for float packages
\usepackage{scrhack}

% -------------------------------------------------------- Compiler detection
\usepackage{iftex}

% ---------------------------------------------------------------------  utf8
\ifpdftex
\usepackage[utf8]{inputenc}
\usepackage[T1]{fontenc}
\fi

% -----------------------------------------------------------------  language
\usepackage{ifthen}
\ifthenelse{\equal{\thesislanguage}{german}}{%
  \usepackage[ngerman]{babel}%
}{}

% --------------------------------------------------------------------  Fonts
\usepackage{amssymb,amsthm,mathtools} % Load before unicode-math (for LuaTeX)
\ifluatex
  \usepackage{fontspec}
  \usepackage{unicode-math}
  \setmainfont[
      Ligatures={Common,Rare,TeX},
      ItalicFeatures={Scale=MatchUppercase,LetterSpace=2.0}
    ]
    {TeX Gyre Pagella}
  \DeclareFontShape{TU}{TeXGyrePagella(0)}{m}{scsl}{<->ssub * TeXGyrePagella(0)/m/scit}{} %hack
  \setmathfont[math-style=ISO,bold-style=ISO]{TeX Gyre Pagella Math}
  \DeclareMathAlphabet{\mathcal}{OMS}{cmsy}{m}{n}
\else\ifpdftex
\usepackage[osf,sc]{mathpazo}
\usepackage{kpfonts}
% \usepackage[scale=.8]{sourcecodepro}
\usepackage[scaled=0.8]{beramono}
\usepackage{microtype}
% \renewcommand*{\ttdefault}{sourcecodepro}
% make texttt never be bold or italic... It is slated for a reason!
% \let\OldTexttt\texttt
% \renewcommand{\texttt}[1]{\textnormal{\OldTexttt{#1}}}
% \KOMAoption{fontsize}{10pt}
% \recalctypearea

% https://tex.stackexchange.com/questions/327184/copy-search-of-old-style-numbers-with-kpfonts
\pdfglyphtounicode{zerooldstyle}{0030}
\pdfglyphtounicode{oneoldstyle}{0031}
\pdfglyphtounicode{twooldstyle}{0032}
\pdfglyphtounicode{threeoldstyle}{0033}
\pdfglyphtounicode{fouroldstyle}{0034}
\pdfglyphtounicode{fiveoldstyle}{0035}
\pdfglyphtounicode{sixoldstyle}{0036}
\pdfglyphtounicode{sevenoldstyle}{0037}
\pdfglyphtounicode{eightoldstyle}{0038}
\pdfglyphtounicode{nineoldstyle}{0039}
\pdfgentounicode=1
\fi\fi
\usepackage{microtype}

% -----------------------------------------------------------------  Spacings
\usepackage{setspace}
% setze den zeilenabstand:
\linespread{1.23}
\setlength{\parskip}{.8em} % Paragraph spacing
% \areaset{43em}{61\baselineskip}

% disables indentation of paragraphs, there is space between them #arnold
% \parindent=0pt


% -------------------------------------------------------------------  Colors
\usepackage[usenames,dvipsnames,svgnames]{xcolor}
\ifthenelse{\equal{\thesismonochrome}{true}}%
    {\definecolor[named]{uboBlue}{cmyk}{0,0,0,1}}%
    {\definecolor[named]{uboBlue}{cmyk}{1,.7,0,0}}

\definecolor[named]{uboYellow}{cmyk}{0,.3,1,0}
\definecolor[named]{uboGrey}{cmyk}{0,0,.15,.55}

% ----------------------------------------------------------------  TOC,TOF,TOL,
\usepackage{listings}
\usepackage[figure,table,lstlisting]{totalcount}
\usepackage{lstautogobble}  % Fix relative indenting

\DeclareTOCStyleEntries[
  raggedentrytext, % titel nich tim blocksatz
  numwidth=7em,
  % entrynumberformat=\raggedleft,
  rightindent=3em,
  pagenumberwidth=2em%,
  % numsep=1em%,
  % dynnumwidth
]{tocline}{figure,table}


\lstset{literate=
  {ö}{{\"o}}1 {ü}{{\"u}}1 {Ä}{{\"A}}1 {ä}{{\"a}}1 {Ö}{{\"O}}1 {Ü}{{\"U}}1
  {ß}{{\ss}}1
}

% ------------------------------------------------------------ Style lstlisting,

\lstset{
  backgroundcolor=\color{white},   % choose the background color; you must add \usepackage{color} or \usepackage{xcolor}; should come as last argument
  basicstyle=\ttfamily,        % the size of the fonts that are used for the code
  breakatwhitespace=false,         % sets if automatic breaks should only happen at whitespace
  breaklines,                      % sets automatic line breaking
  breakautoindent,
  %postbreak=.,
  autogobble,
  captionpos=b,                    % sets the caption-position to bottom
  commentstyle=\itshape\color{gray}, %\color{mygreen},    % comment style
  deletekeywords={...},            % if you want to delete keywords from the given language
  escapeinside={\%*}{*)},          % if you want to add LaTeX within your code
%  extendedchars=true,              % lets you use non-ASCII characters; for 8-bits encodings only, does not work with UTF-8
  frame=none,	                   % adds a frame around the code
  %framesep=6pt,
  keepspaces=true,                 % keeps spaces in text, useful for keeping indentation of code (possibly needs columns=flexible)
  keywordstyle=\color{uboBlue}\bfseries,       % keyword style
  language=C,                 % the language of the code
  morekeywords={*,...},            % if you want to add more keywords to the set
  numbers=left,                    % where to put the line-numbers; possible values are (none, left, right)
  numberstyle=\scriptsize, % the style that is used for the line-numbers
  rulecolor=\color{uboBlue},         % if not set, the frame-color may be changed on line-breaks within not-black text (e.g. comments (green here))
  showspaces=false,                % show spaces everywhere adding particular underscores; it overrides 'showstringspaces'
  showstringspaces=false,          % underline spaces within strings only
  showtabs=false,                  % show tabs within strings adding particular underscores
  stepnumber=1,                    % the step between two line-numbers. If it's 1, each line will be numbered
%  stringstyle=\color{mymauve},     % string literal style
  tabsize=2,	                   % sets default tabsize to 2 spaces
%  title=\caption,                  % show the filename of files included with \lstinputlisting; also try caption instead of title
%  framexleftmargin=1.5em,         % framelines over line numbers
  xleftmargin=2.5em                  % fit line numbers in textwidth
}


% -------------------------------------------------------------------  Titles
\setkomafont{chapter}{\normalfont\bfseries\huge\scshape\color{uboBlue}}
\renewcommand\chapterformat{\normalfont\Large\bfseries\color{black}\thechapter\hspace{1.4ex}}

\setkomafont{section}{\normalfont\bfseries\Large\scshape\color{uboBlue}}
\renewcommand\sectionformat{\normalfont\large\bfseries\color{black}\thesection~}


\setkomafont{subsection}{\normalfont\bfseries\large\scshape\color{uboBlue}}
\renewcommand\subsectionformat{\normalfont\normalsize\bfseries\color{black}\thesubsection~}


\setkomafont{subsubsection}{\normalfont\bfseries\normalsize\scshape\color{uboBlue}}
\renewcommand\subsubsectionformat{\normalfont\small\bfseries\color{black}\thesubsubsection~}

\setkomafont{paragraph}{\normalfont\bfseries\normalsize\scshape\color{uboBlue}}
\renewcommand\paragraphformat{\normalfont\small\bfseries\color{black}\theparagraph~}


\RedeclareSectionCommand[
    beforeskip=4\baselineskip,
    afterskip=3\baselineskip]{chapter}
\RedeclareSectionCommand[
    beforeskip=-1\baselineskip,
    afterskip=1pt]{section}
\RedeclareSectionCommand[
    beforeskip=-.75\baselineskip,
    afterskip=1pt]{subsection}
\RedeclareSectionCommand[
    beforeskip=-.5\baselineskip,
    afterskip=1pt]{subsubsection}

% change TOC accordingly
\setkomafont{chapterentry}{\normalfont\bfseries\scshape}

\RedeclareSectionCommand[
  beforeskip=-\baselineskip,
  afterskip=2\baselineskip]{chapter}
% issue 59: we cannot use beforeskip with a non-negative value as it changes indentation behavior
\renewcommand*{\chapterheadstartvskip}{\vspace*{3\baselineskip}}
\RedeclareSectionCommand[
  beforeskip=-\baselineskip,
  afterskip=.2\baselineskip]{section}

% --------------------------------------------------------------------  Lists
\usepackage{enumitem}
\setlist{itemsep=-5pt}
\newcommand{\localtextbulletone}{\textcolor{uboBlue}{{\tiny$\blacksquare$} }}
\renewcommand{\labelitemi}{\localtextbulletone}
\setlist[enumerate]{label=\textbf{\color{uboBlue}{\arabic*.}}}
\setlist[description]{font=\normalfont\scshape\bfseries\textcolor{uboBlue},itemsep=2pt}

% -----------------------------------------------  captions, figures & tables
\usepackage{chngcntr}
\counterwithout{figure}{chapter}
\counterwithout{figure}{section}
\counterwithout{table}{chapter}
\usepackage{subfigure}
\setkomafont{captionlabel}{\normalfont\scshape\bfseries\textcolor{uboBlue}}
\setkomafont{caption}{\normalfont\itshape}

% tables
\usepackage{floatrow}
\floatsetup[table]{capposition=top}
\usepackage{booktabs}
\newcommand{\tabhead}[1]{\textsc{\textbf{#1}}}
% ----------------------------------------------------------  PDF & Hyperrefs
% Note: hyperref should be "last package"
\usepackage{graphicx}
\usepackage{url}
% Allow to break links after / and after -
\def\UrlBreaks{\do\/\do-}

% -------------------------------------------------  Headers & Page Numbering
\usepackage[automark]{scrlayer-scrpage}
\clearpairofpagestyles
\automark[section]{chapter}
\ohead{\scshape\rightmark}
\ofoot[\scshape\pagemark]{\scshape\pagemark}

% --------------------------------------------------------------  Definitions
\newtheoremstyle{thesisdefinition}%
    {\topsep}%
    {\topsep}%
    {\normalfont}%
    {0ex}%
    {\scshape\bfseries\color{uboBlue}}%
    {:}%
    {1ex}%
    {}
\newtheoremstyle{thesisplain}%
    {\topsep}%
    {\topsep}%
    {\normalfont\itshape}%
    {0ex}%
    {\scshape\bfseries\color{uboBlue}}%
    {:}%
    {1ex}%
    {}
\newtheoremstyle{thesisremark}%
    {\topsep}%
    {\topsep}%
    {\normalfont}%
    {0ex}%
    {\scshape\itshape\color{uboBlue}}%
    {:}%
    {1ex}%
    {}

% Definitions
\theoremstyle{thesisdefinition}
\newtheorem{definition}{Definition}
\newtheorem*{definition*}{Definition}
% Theorems
\theoremstyle{thesisplain}
\newtheorem{theorem}[definition]{Theorem}
\newtheorem*{theorem*}{Theorem}
\newtheorem{proposition}[definition]{Proposition}
\newtheorem*{proposition*}{Proposition}
\newtheorem{lemma}[definition]{Lemma}
\newtheorem*{lemma*}{Lemma}
% Remarks
\theoremstyle{thesisremark}
\newtheorem{remark}[definition]{Remark}
\newtheorem*{remark*}{Remark}

% -------------------------------------------------------------------  Quotes
\ifthenelse{\equal{\thesislanguage}{german}}{%
  \usepackage[german=quotes]{csquotes}%
}{%
  \usepackage{csquotes}%
}
% format bigger quotes
\renewcommand{\mkcitation}[1]{\space#1}
\renewcommand{\mkblockquote}[4]{\openautoquote#1\closeautoquote#2#4#3}

\newcommand{\smallquote}[2]{%
  \emph{»#1«} #2
}
\def\signed #1{{\leavevmode\unskip\nobreak\hfil\penalty50\hskip2em%
  \hbox{}\hfil#1%
  \parfillskip=0pt \finalhyphendemerits=0 \endgraf}}
\newsavebox\mybox
\newenvironment{bigquote}[1]
  {\savebox\mybox{#1}\begin{quote}\itshape»}%
  {«\normalfont\signed{\usebox\mybox}\end{quote}}

% -----------------------------------------------------------------  Algorithms
\ifthenelse{\equal{\thesislanguage}{german}}{%
  \usepackage[linesnumbered,lined,ngerman,german,figure]{algorithm2e}%
}{%
  \usepackage[linesnumbered,lined,figure]{algorithm2e}%
}%
\setlength{\algomargin}{4em}
\SetAlgoSkip{bigskip}
\SetAlgoInsideSkip{medskip}
% -------------------------------------------------------------------  Examples
\usepackage{tcolorbox}
\newenvironment{example}{}{}
\tcbuselibrary{skins}
\tcbuselibrary{breakable}
\tcolorboxenvironment{example}{blanker,breakable,left=5mm,%
before skip=10pt,after skip=10pt,borderline west={1.2mm}{0pt}{uboBlue!50!white}}

% ----------------------------------------------- Restrict widows and orphans
\clubpenalty=10000
\widowpenalty=10000
\displaywidowpenalty=10000
% ----------------------------------------------------------------- Marginpar
\let\oldmarginpar\marginpar
\renewcommand{\marginpar}[1]{%
    \oldmarginpar[\raggedleft\em\color{Grey}#1]{\raggedright\em\color{Grey}#1}
}

% ----------------------------------------------------------------- Bibliography
\ifthenelse{\equal{\thesisreferencestyle}{number}}{%
  \usepackage[backend=biber,
		style=numeric,
		isbn=false,
    maxbibnames=9,
    maxcitenames=3,
		doi=false]{biblatex}%
}{
  \usepackage[backend=biber,
		style=alphabetic,
		isbn=false,
		doi=false,
    maxbibnames=9,
    maxcitenames=3
		]{biblatex}%
}
\renewcommand{\labelnamepunct}{\addcolon\space}
\DeclareNameAlias{author}{family-given}
\AtBeginBibliography{\renewcommand*{\mkbibnamefamily}[1]{\textsc{#1}}}
\AtEveryBibitem{%
   \clearfield{day}%
   \clearfield{month}%
   \clearfield{endday}%
   \clearfield{endmonth}%
}
\ifthenelse{\equal{\thesislanguage}{german}}{%
    \newcommand{\reftitle}{Literaturverzeichnis}
}{%
    \newcommand{\reftitle}{References}
}%
\DefineBibliographyStrings{german}{
  mathesis = {Masterarbeit},
}


\addbibresource{tpl/rfc.bib}

% break urls freaggin everywhere in the bibliography!
\setcounter{biburllcpenalty}{1000}
\setcounter{biburlucpenalty}{1000}
\setcounter{biburlnumpenalty}{1000}

\renewcommand{\finalnamedelim}{\addspace\&\space}
\AtBeginBibliography{%
  \renewcommand{\multinamedelim}{\addsemicolon\space}%
}


% https://tex.stackexchange.com/a/66184/229658
\renewbibmacro*{issue+date}{%
  \setunit{\addcomma\space}% NEW
%  \printtext[parens]{% DELETED
    \iffieldundef{issue}
      {\usebibmacro{date}}
      {\printfield{issue}%
       \setunit*{\addspace}%
%       \usebibmacro{date}}}% DELETED
       \usebibmacro{date}}% NEW
  \newunit}

% ------------------------------------------------- hyperref should be "last package"
\usepackage{hyperref}
\hypersetup{
    colorlinks=true,
    unicode=true,
    pdftitle={\thesistitle},
    pdfauthor={\thesisauthor},
    pdfsubject={\thesissubtitle},
    urlcolor=uboBlue,
    linkcolor=black,
    citecolor=uboBlue
}
\usepackage[all]{hypcap}        % Convenience: Let hyperref jump to the figure instead to the caption only.
\ifthenelse{\equal{\thesislanguage}{german}}{%
  \usepackage[ngerman]{varioref}%
  \usepackage[ngerman,noabbrev,capitalise]{cleveref}%
}{%
  \usepackage{varioref}%
  \usepackage[noabbrev,capitalise]{cleveref}%
}

% -------------------------------------------------- Glossary
\usepackage[abbreviations,nonumberlist]{glossaries-extra}

% make first appearance emphasized
\renewcommand{\glsfirstlongdefaultfont}[1]{\emph{#1}}
\renewcommand{\glsfirstabbrvdefaultfont}[1]{\emph{#1}}

% deactivate hyperlinks to glossary if it will not be printed
\ifthenelse{\equal{\thesisprintglossary}{true}}{
  \makenoidxglossaries
}{
  \glsdisablehyper
}
